\section*{Experimental part}
\label{sec:expr}

Our analysis tool was conceived using the \verb|Python| programming language,
version \verb|3.6.3|. It consists of a main component, the analyser
(\verb|analyser.py|), and a pattern module (\verb|pattern.py|) that houses the
\verb|Pattern| class, used to instantiate objects that represent vulnerable
patterns.

\subsection*{Analyser}
\label{sec:expr.analyser}

The main component is run by invoking it and passing it a PHP program slice in
JSON format as a command line argument.

\begin{verbatim}
    analyser.py /path/to/slice.json
\end{verbatim}

The slice is already in the form of an AST, according to the syntax of the AST's
generated by Glayzzle's PHP Parser \cite{glayzzle-php}.

\subsection*{Patterns}
\label{sec:expr.patterns}

The \verb|patterns| file contains a set of vulnerable patterns with the
following format:

\begin{lstlisting}[label={lst:pat.tmpl},
        caption={Vulnerable pattern template}]
    Vulnerability
    Entry|\tsub{1}|,Entry|\tsub{2}|,...,Entry|\tsub{i}|
    Sanitizer|\tsub{1}|,Sanitizer|\tsub{2}|,...,Sanitizer|\tsub{j}|
    Sink|\tsub{1}|,Sink|\tsub{2}|,...,Sink|\tsub{k}|
\end{lstlisting}

where \verb|Vulnerability| is the name of the vulnerability, \verb|Entry| is an
entry point, \verb|Sanitizer| is a sanitization/validation function, and
\verb|Sink| a sensitive sink. Example:

\begin{lstlisting}[label={lst:pat.ex},
        caption={SQL Injection pattern, specific to PostgreSQL}]
    |\label{lst:lst1}|
    SQL injection (PostgreSQL)
    $_GET,$_POST,$_COOKIE,$_REQUEST
    pg_escape_string,pg_escape_bytea
    pg_query,pg_send_query
\end{lstlisting}