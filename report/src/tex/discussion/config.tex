\subsection*{Tool Configurability and Range}
\label{sec:discussion.config}

Despite being able to analyse the slices given to us as an example to base the
tool around, it doesn't go far beyond it. The amount of language elements and
scenarios considered is pretty confined to the ones present in the example
slices.

In spite of these restrictions, adding language elements isn't as difficult as
it seems, though, likewise, not as easy as it could have been made to be. Since
the analysis is broken down into various functions, each attributed to a
language element (i.e. a node in the AST), meaning, if a language element is to
be added, implementing the functions for the nodes being introduced should be
sufficient. With it, taint analysis is also a requirement, meaning the tracking
of tainted objects/symbols must be implemented along with the basic visiting
model.

On the up side, the addition of patterns, if need be, is imensely easy, since
all that is needed to do is add it to the \verb|patterns| file in the format 
seen in Listing~\ref{lst:pattern.template} and the analyser will also consider
it accordingly. The current set of patterns was in part retrieved from
\cite{wap-manual:2017}.
