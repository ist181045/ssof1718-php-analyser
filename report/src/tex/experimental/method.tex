\subsection*{Method}
\label{sec:experimental.method}

We approached the problem in a somewhat simplistic and restrictive way, meaning,
unfortunately, that achieving modifiability is hard, yet not impossible, due to
the limited subset of language elements considered and the rudimentary way the
tool was designed.

To start off, the patterns are fetched from the \verb|patterns| file and parsed,
generating a list of \verb|Pattern| objects.

Then, the JSON formatted slice is loaded and the AST is converted into a Python
dictionary, which is used throughout the analysis. Adopting a visitor-like
model, the tool's able to analyse the nodes separately, which seemingly contrary
to what was previously stated, does enable change, in the sense that for another
construct to be introduced, one or two function need(s) to be added to analyse
the corresponding node.

During the traversal, a list of tainted symbols (variables) is carried along, as
well as a dictionary of variables and their respective values. This allows us to
perform, as we go, some basic taint analysis as we can detect when a tainted
object is used in a sensitive sink.

Once such a case is detected, the program reports the vulnerability, suggests
a set of possible sanitizations that can be used and, for simplicity's sake,
exits, since we assumed not more than one vulnerability was present in the given
slices of code.
